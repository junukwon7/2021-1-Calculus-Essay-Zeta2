\begin{abstract}
The discussion topic in the Math Seminar I class was the examination of  $\sum_{n=1}^\infty 1/n^2$. In the class, we solved it using the Fourier series of $f(x)=x$. However, the teacher said there are numerous proofs out there which uses brilliant techniques so I searched for them and found out that they use Calculus-techniques like integration and Taylor expansion. So, I kept the topic to study about it later.

This report will suggest various ways to prove the identity $\zeta(2) = \sum_{n=1}^\infty 1/n^2 = \pi^2/6$, which is also as known as the Riemann function at 2 or the Basel problem. Also, the report will figure out important theorems and proofs needed while proving the identity. ~\\


수학세미나I 수업 시간에 $\sum_{n=1}^\infty 1/n^2$의 값을 구하는 토의 학습을 했다. 수업에서는 $f(x)=x$의 푸리에 급수를 이용해 값을 구할 수 있었다. 선생님께서는 다른 획기적인 풀이들도 많다고 하셨기에 찾아봤고, 대부분의 풀이들이 적분이나 테일러 급수와 같은 미적분학 이론을 사용한다는 것을 알게 되었다. 이후 미적분학I 수업을 들으며 얻은 지식들을 토대로 위 문제의 답을 구하는 여러 방법을 제시하고자 한다.

본 보고서는 리만 제타함수, 그리고 p급수의 수렴판정과 깊은 관련을 가진 항등식이자 바젤 문제로도 잘 알려진 식인 $\zeta(2) = \sum_{n=1}^\infty 1/n^2 = \pi^2/6$을 증명하는 다양한 방법에 관해 기술하고 있다. 또한, 항등식을 보이는 과정에서 중요한 정리 및 증명들을 짚고 넘어갈 것이다.~\\\\
\end{abstract}
\begin{dedication2}
{\large{19007 권준우}}\\[5mm]
2021년 미적분학I 5반
\end{dedication2}
\begin{copyright2}
	
	\begin{figure}[h]
		\includegraphics[width=3in]{logo_full}
	\end{figure}
	
	\noindent Copyright \textcopyright\the\year\ Junu Kwon \\ % Copyright of the form
	
	\noindent \textsc{https://www.gs.hs.kr/}\\ % URL
	
	\noindent \textit{Compiled \today} 
	\if@openright\cleardoublepage\else\clearpage\fi
\end{copyright2}